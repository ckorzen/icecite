\section{System infrastructure}
In the sections \ref{sec:extraction} and \ref{sec:identification:references} we have already introduced two of the main components of Icecite, namely the automatic identification of metadata and references from research papers. In this section, we will give you a general overview of the implementation details of all further components. Moreover, we will discuss, how the components act together and how they are embedded into the typical workflows of Icecite.

\subsection{The User Interface}
The user interface of Icecite is fully web-based and was developed with GWT (\textit{Google Web Toolkit}). It is mainly splitted into the library view (shown in figure \ref{fig:screenshot_libraryview}) and into the document view (shown in figure \ref{fig:screenshot_documentview}). In the library view, each document of the personal library is displayed with its full metadata. In the document view, the PDF file is placed next to the identified references of a selected document. To add a document to the library, there are generally three ways:

\par\medskip\noindent
\textit{(ADD 1)} Upload a PDF file to the library.
\par\medskip\noindent
\textit{(ADD 2)} Click an entry in a reference list in the document view.
\par\medskip\noindent
\textit{(ADD 3)} Type a search query into the search field of the library view to browse for any entries in a digital library (DBLP or Medline). Click a record of the search result.
\medskip

Variant \textit{(ADD 1)} is followed directly by the automatic identification of the metadata and the bibliographic references of the research paper given by the uploaded PDF file. Once all the metadata and references are retrieved, the document is added to the library.

In contrast, variants \textit{(ADD 2)} and \textit{(ADD 3)} are firstly followed by the automatic search for a PDF file, that contains the fulltext of the clicked record in the references list (resp. the search result). Once the PDF file was found, the further processing is similar to that for variant \textit{(ADD 1)} as described above. The automatic search for PDF files is discussed in detail in section \ref{sec:search}.

In the document view, PDF files can be read if there is a proper browser plugin is installed to display the PDF files in the browser. It can be even annotated if there is the plugin of Adobe Acrobat is installed. On implementing this feature, attention was paid to use \textit{native}\footnote{With native annotations we mean those annotations, which conform to the official PDF specifications} annotations. This ensures, that the annotations are identified as such not only within Icecite, but also in various external PDF viewers. See the detailed description of this feature in section \ref{sec:annotations}, why this intention isn't trivial to achieve.

All library data (metadata, PDF files, etc.) are stored locally on client's filesystem via the HTML5 FileSystem API, such that the basic features of Icecite can be even used offline. To keep the data consistent with the server and to share the data with other users, the data are synchronized periodically with the server. See section \ref{sec:sync} for a detailed discussion of the data synchronization.

\subsection{The search functionality}\label{sec:search}
Obviously, a PDF file is needed to identify the metadata and the bibliographic references for a research paper. However, there are no PDF files given on adding documents to the library via the methods \textit{(ADD 2)} or \textit{(ADD 3)}. That's why Icecite is able to find PDF files automatically in web. In this section, we describe both, the basic search functionality to browse the own library and external digital libraries and the functionality to search for PDF files automatically.

The core of Icecite's search functionality is built with \textit{CompleteSearch}, an interactive and efficient search engine, implemented by Bast and Weber \cite{DBLP:conf/cidr/BastW07}. On typing a search query in search field of the library view, all records in DBLP and Medline are searched on the one hand. On the other hand, also the metadata, the fulltext and the annotations (i.e. comments and tags, see section \ref{sec:sync}) of each document in the personal library are searched. Hence, there are two instances of \textit{CompleteSearch}, each with different underlying indexes: one containing the data of DBLP and Medline($I_{DM}$) and one containing the data of personal library ($I_L$). Consider, that the data of library are quite volatile and that modifications on it are very common. So to keep the index $I_L$ up to date, the indexing is triggered automatically on every modification in library. Apart from that, the data in DBLP and Medline are more settled, so that the update of index $I_{DM}$ is necessary less frequently and is triggered manually from time to time.

The search results from both indexes are combined and displayed in library view, where the resulting entries of $I_L$ are listed before the entries of $I_{DM}$. The entries of $I_L$ are enriched with fulltext excerpts to highlight the matching parts in fulltext. Clicking an entry of $I_L$ triggers a forwarding to the \textit{document view} and clicking an entry of $I_{DM}$ induces the adding of the entry to the library and hence the automatic search for an accordant PDF file.

The process of the searching a PDF file automatically is multistage. At first stage, Icecite follows up an url, that is provided by DBLP \TODO{Add Medline}. Ideally, this url (called "ee" in DBLP, ee = electronic edition), refers to the location of the PDF file. In this case, the searched PDF file can be downloaded from the ee. However, the url usually doesn't refer to the PDF file directly, but to any document-specific page of the publisher, where the download of the PDF file is usually restricted by the publisher and only available for licensees.

Icecite checks the availability of the PDF file and downloads it, if applicable. If there is no such PDF file or access to it is restricted, Icecite switches to the second stage, where it searches for the file by querying Google with the title and the author names of the record. \TODO{Is it appropriated to mention this here? (because its not legal)}
Nevertheless, Icecite isn't always able to identify the correct PDF file among Google's search result definitely. Thus, a set of file candidates are shown to the user, where he can choose the correct one.

\subsection{Synchronizing and sharing data} \label{sec:html5} \label{sec:sync}
In Icecite, all data of the personal library (metadata of documents, PDF files, annotations of PDF files, etc.) are stored locally on the client utilizing the HTML5 \textit{FileSystem API}. This API allows a web application to read and write data to a sandboxed section of the local file system. Once the library data are requested from the server for the first time, they will be stored locally where they can be browsed without requesting them from the server again and again. In combination with another HTML5 feature called \textit{ApplicationCache}, some features of Icecite are even accessible in offline scenarios, for example: Browsing the library, uploading PDF files into the library, reading and annotating PDF files in browser. 

To keep the locally stored data consistent with the server, the data are synchronized periodically. Due to the synchronization, documents of the personal library are shareable with other users. A major benefit is to be able to see all annotations made by various users in a shared PDF file nearly in real time. The background of the synchronization and sharing process is built by a pure \textit{Subversion} system. It is utilized to manage the individual data sets of users on the server and to combine the individual data sets of shared libraries. Due to Subversion, all changes on the datasets are fully  reproducible and any conflicts (e.g. on annotations) are generically resolvable.

\subsection{Annotating PDF files in browser} \label{sec:annotations} 
During the implementation of a feature to be able to annotate PDF files in browser, we paid special attention to utilize \textit{native} annotations in PDF files. Hence, annotations are identified as such not only within Icecite, but also in the most conventional external PDF viewers. Moreover, annotations aren't embedded immutably into PDF files in this way -- instead, they are modifiable even in external PDF viewers. \TODO{Mention here, that other applications don't use native annotations}. 

All annotations data are stored separately from the PDF files, such that Icecite is able to synchronize the annotations independently from the PDF files. So, the annotations must not always be transferred to server on synchronizing changes on annotations. Because the annotations aren't stored within a PDF file, they are injected instantly when opening the file.

The mentioned characteristics were implemented against the widespread Adobe Acrobat Standard plugin. This plugin are aimed to display PDF files in a web browser. However, PDF files cannot be annotated in a browser out of the box, even if it's feasible in the desktop application of Adobe Acrobat. In the following we present a method, how PDF files can be annotated in the browser nonetheless.

Generally, a PDF file can include several pieces of javascript code, that is triggered on certain events, for example on opening the PDF file. Accordingly, a small javascript snippet is injected into a PDF file when adding it to the library. This snippet unlocks the standard annotation tools of the Adobe Acrobat plugin which are locked per default \TODO{why?}. Furthermore, it establishes a message handler, that allows a two-way communication between a PDF file ($P$) and the surrounding web application ($A$). This message handler is used in the direction $A \rightarrow P$ (1) to send the annotations to inject on opening a PDF file and (2) send the annotations to add/edit/delete after a synchronization process. In contrast, the direction $P \rightarrow A$ is used to inform the application about user actions regarding the annotations (e.g. add, edit, delete an annotation).