\section{Metadata identification}\label{sec:extraction} 
In this section, we describe our approach to identify the metadata of research papers in detail. For a discussion of the bibliographic references identification, see section \ref{sec:identification:references}. As mentioned above, we use a simple rule-based algorithm to extract the title from the PDF file of a research paper. After the extraction, the title is matched to a record of the metadata knowledge base (which is given by DBLP and Medline) to get complete and correct metadata. The algorithm to extract the titles from research papers is described in section \ref{sec:trextraction}, whereas the matching process is described in \ref{sec:trmatching}.

\subsection{The extraction of titles from PDF files}\label{sec:trextraction}
Basically, we use the open source Java library \textit{PDFBox}\footnote{http://pdfbox.apache.org/} to extract data from PDF files. With PDFBox, we get basically the location, the width, the height and the font) of each character. Let these attributes describe the \textit{type} of an character. With these types, the types of all words and all text lines are reconstructed. Furthermore, all text lines are segmented into several regions such that each region contains all consecutive lines with the same type. Figure \ref{fig:division_regions} shows an example how such a division might look like. It results a natural text hierarchy with the following layers: 
$$ characters \rightarrow words \rightarrow text lines \rightarrow regions $$
Each layer offers its own type, whereas the type of a region follows by the types of the contained text lines; the type of each text line follows by the type of each contained word, and so on. Based on the types, we define a natural order on the regions, which is given by their font sizes followed by their font styles. On sorting by font styles, the following order is applied: $$\textbf{bold} > \textit{italic} > \text{normal}.$$
Finally, all types with the same font sizes and the same font styles are ordered by their locations.

\begin{figure}
\centering
\fbox{\includegraphics[page=1,viewport=25mm 160mm 195mm 255mm,clip,width=0.97\columnwidth]{./figures/division_regions}}      
\caption{Example for a division of a PDF file into regions.}
\label{fig:division_regions}
\end{figure}

On identifying a title, we assume that all lines belonging to the title are placed immediately after each other on the front page. Further we assume, that these lines possess the same type and that this type is the \textit{largest}  one (according to their defined order) compared to the types of the remaining text lines. It follows from the assumptions, that there is exactly one region placed on the front page, that contains the title exclusively. It possesses the largest type and so, it can be identified by sorting the regions by their types.

\subsection{The matching of extracted titles}\label{sec:trmatching}

Once the title of a research paper is extracted, it is matched to a record of the metadata knowledge base ($KB$) to get full and correct metadata. The basics of the matching process is quite simple and base on computing the Levenshtein distances between the extracted title and the titles of $KB$ records.
However, as mentioned in section \ref{sec:relatedwork:recordmatching}, brute-force pairwise comparisons over all $KB$ records are not feasible. That's why the matching process is preceded by a weak filtering mechanism to get a small-sized set of candidates $C$ from the $KB$. The costly computations of the Levenshtein distances are then only executed on the set of candidates.
   
The filtering mechanism base on an ordinary \textit{inverted index} $I$. The index is filled with all words, which results from normalizing the title, author names and years of each record in the $KB$ (author names and years are included for the references matching process, which is described in section \ref{sec:matching:references}). Normalizing a string means:
\begin{enumerate}
\renewcommand{\labelenumi}{(\arabic{enumi})}
\setlength{\itemsep}{-1pt} 
\item Split the string into words, 
\item Transform each character into lowercase,
\item Remove all punctuations, special characters and stop words. 
\end{enumerate}
With the inverted index, each word $w$ is mapped to an inverted list $I_w$ containing all the $KB$ records, whose metadata contain the word.
 
The set of candidates are queried from the index in a typical manner: First, the extracted title is also normalized (as described above) and for each resulting word the associated inverted list of records are fetched. In a second step, all the lists are merged into a single list $L$. Its elements are sorted by their occurrences in the fetched lists (and hence by the number of shared words between the record and the extracted title). $C$ is then given by the top-$k$ elements of $L$, where $k>0$ is an arbitrary, but fixed integer value. The subsequent computations of the Levenshtein distances are restricted to $C$ and finally, the extracted title is matched to the record, for which the Levenshtein distance between its title and the extracted title is minimal.

\section{References identification} \label{sec:identification:references}
In this section, we describe our approach to identify the bibliographic references of research papers in detail. Again, a rule-based algorithm is used to extract the references from PDF files and each extract is matched to a record of the metadata knowledge base. The algorithm to extract the references is described in section \ref{sec:trextraction} and the matching process is described in \ref{sec:trmatching}.

\subsection{The extraction of references from PDF files}
Compared to the extraction of the title, the extraction of references is much more challenging. That's because there isn't a universal convention on designing a reference or on arranging its metadata fields. So we have to consider various styles for the bibliography entries. 

The text lines of a PDF file are extracted on the same way as described for the title extraction. To identify the position of the bibliography in the research paper, all the lines are checked for the bibliography header, e.g. by analyzing their types and by searching for the words \textit{References}, \textit{Literature}, \textit{Bibliography}, etc. 
On the references identification, only text lines are considered, which follows the bibliography header and which are members of the so called \textit{content box}. The \textit{content box} of a page contains all the \textit{relevant} text lines in a page, that means all text 
lines except the page headers and the page footers. 

These text lines are traversed sequentially. On visiting a text line, also the previous and the next line are examined to mark the with one of the following labels:

\newcommand{\refhead}{\textit{($R_H$)} }
\newcommand{\refbody}{\textit{($R_B$)} }
\newcommand{\bibend}{\textit{($E$)} }

\par\medskip\noindent
\refhead line is a reference header (the first line of a reference)
\par\smallskip\noindent
\refbody line is a member of the reference body (all further lines of a reference)
\par\smallskip\noindent
\bibend line is the last line in the bibliography.
\medskip

\noindent Consider, that each reference consists of exactly one line, which is marked with \refhead and of an arbitrary number of lines, which are marked with \refbody. The extraction process is finished, if a line was marked with \bibend. Hence, to extract the references, all text lines have to be marked correctly.

Depending on the layout of a bibliography and its references, marking a line with the correct label may be a non-trivial task. However identifying the end of a bibliography is comparatively simple: a line is marked with \bibend, if the type of the next line is larger than the type of the current line or if the line pitch between the current line and the next line is "too" large. But especially the distinction between reference headers and reference bodies is challenging. The basic idea of our approach is to identify characteristic attributes, on which a label of a line can be reasoned. Examples for such attributes are discussed in the following.

\paragraph{The existence of reference anchors}
\TODO{Assume, that the attributes are consistent}
\TODO{Example}
\textit{Reference Anchors} are unique and usually short identifiers, which are  
prepended to reference headers, like \texttt{[1]}, \texttt{[Miller2012]}, etc.
If the reference headers of a bibliography contains reference anchors, marking the text lines is simple: If a line is prepended with an anchor, it is labeled with \refhead and otherwise with \refbody. 

\paragraph{The indentation of reference bodies}
\TODO{Example}
\TODO{Pseudocode?}
Another characteristic attribute is the indentation of the reference bodies compared to the reference headers. In this case, marking the lines is a bit more tricky: Initially, the first line (after the header) of the bibliography is labeled with \refhead. All further lines are marked as follows: If a line is intended to either the previous or the next line, it is marked as \refbody.
If the previous or the next line is intended to the current line, it is marked as \refhead. Otherwise, it is marked with the label of the previous line.

Special care is needed, if the text lines are arranged in multiple columns and the column of the previous or the next line differs from the column of the current line. Therefore, it must be taken into account on checking for indentations, that one of the lines is shifted horizontally by a specific amount.

\paragraph{The line pitches between references}
If no attributes introduced so far are used in a bibliography, marking the lines is most difficult. Another but weak characteristic attribute are the line pitches between each individual reference, which are usually larger than the line pitches between two common text lines. Again, the first line of the bibliography is marked with \refhead in this case. Further lines are also marked with \refhead if its pitch to the previous line is larger than the common one. Consequently, a line is marked with \refbody it its pitch correlates with the common line pitch. Obviously, marking a line with \refhead or \refbody is challenging if the previous line is located in another column or another page.

\subsection{The matching of extracted references} \label{sec:matching:references}
The process of matching an extracted reference string is quite similar to that of matching an extracted title. Thus, a set of candidates are determined for an extracted reference in the same way. However the processes differ in the evaluation of these candidates. Because a reference doesn't contain only the title, but also further metadata of the belonging research paper (e.g. the name of authors, the year of publication, the name of conference, etc.) in unknown order, a different scoring scheme on evaluating the candidates is needed.

We can't discuss the applied scoring scheme in all details here. In principle, it consists of three different similarity scores resulting from the following alignments: 

\par\medskip\noindent
\textit{(A1)} Alignment of the candidates' title against the extracted reference string.
\par\smallskip\noindent
\textit{(A2)} Alignment of the candidates' author names against the extracted reference string.
\par\smallskip\noindent
\textit{(A3)} Alignment of the candidates' year against the extracted reference string.
\medskip

For example, \textit{(A1)} bases on the Smith-Waterman distance, introduced in section \ref{sec:relatedwork:recordmatching}, to find the longest common substring between the extracted reference and the title of a candidate.
Finally, all scores are combined into a single score and the extracted reference is matched to the record, for which this score is maximal. 